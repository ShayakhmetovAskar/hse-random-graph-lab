% encoding: utf-8
\documentclass[12pt,a4paper]{article}

% поддержка русского
\usepackage[T2A]{fontenc}
\usepackage[utf8]{inputenc}
\usepackage[russian]{babel}
\usepackage{cmap}            % копирование кириллицы из PDF

% математические пакеты
\usepackage{amsmath,amssymb}
\usepackage[fleqn]{amsmath}  % выравнивание формул слева

% графика
\usepackage{graphicx}
\graphicspath{{./}}

% гиперссылки
\usepackage[hidelinks]{hyperref}

% отступ первой строки абзаца
\setlength{\parindent}{1em}
% расстояние между абзацами
\setlength{\parskip}{0.5em}

% заголовок
\title{Отчет по части I}
\author{Шаяхметов Аскар}
\date{} % убрать дату

\begin{document}

\maketitle

\section*{Часть 1. Зависимость от параметров распределений}
\vspace{-1em}
\noindent
Значения на графиках — это среднее по $M=100$ независимым реализациям для каждого набора параметров. Число вершин графа $n=100$, параметр $k=5$ для kNN-графа и порог $d=1$ для DIST-графа.

\begin{enumerate}
  \item \textbf{kNN-граф:} Среднее число компонент связности практически не зависит от параметра $\alpha$ SkewNormal (почти горизонтальная кривая около $6$–$6.5$). Для Student-t с ростом $\nu$ число компонент убывает, то есть при «тяжёлых хвостах» ($\nu$ - меньше) граф рассоединен сильнее.

  \item \textbf{DIST-граф:} Среднее кликовое число минимально при $\alpha=0$ и симметрично растёт при удалении от нуля (от $\sim40$ до $\sim70$). Для Student-t кликовое число увеличивается с $\nu$ (от $\sim30$ при $\nu\approx1$ до $\sim40$–$45$ при $\nu\approx10$).
\end{enumerate}

\begin{center}
  \includegraphics[width=0.9\linewidth]{part1_results.png}
\end{center}
\newpage
\section*{Часть 2. Зависимость от $n$, $k$ и $d$}
\vspace{-1em}
\noindent
Значения на графиках — это среднее по $M=100$ независимым реализациям для каждого набора параметров.

\begin{itemize}
  \item \textbf{kNN-граф:}
    \begin{itemize}
      \item При увеличении числа вершин \(n\) (при \(\alpha=\alpha_0\), \(\nu=\nu_0\), \(k=5\)) среднее число компонент связности возрастает.
      \item При увеличении числа соседей \(k\) (при \(\alpha=\alpha_0\), \(\nu=\nu_0\), \(n=100\)) число компонент резко убывает.
    \end{itemize}

  \item \textbf{DIST-граф:}
    \begin{itemize}
      \item При увеличении числа вершин \(n\) (при \(\alpha=\alpha_0\), \(\nu=\nu_0\), \(d=1\)) среднее кликовое число растёт, причём скорость роста выше для SkewNormal-графов.
      \item При увеличении \(d\) (при \(\alpha=\alpha_0\), \(\nu=\nu_0\), \(n=100\)) кликовое число также увеличивается, и для SkewNormal-графов этот рост быстрее. Рост вызван тем, что точки чаще попадают в радиус \(d\).
    \end{itemize}
\end{itemize}


\begin{center}
  \includegraphics[width=0.9\linewidth]{part2_results.png}
\end{center}

\section*{Часть 3. Разделяющая способность статистик}
\vspace{-1em}
\noindent
Построено по $M_{\text{large}}=5000$ реализаций каждого распределения.

\begin{center}
  \includegraphics[width=0.8\linewidth]{part3_results_0.png}
\end{center}
\begin{center}
  \includegraphics[width=\linewidth]{part3_results_1.png}
\end{center}

\begin{itemize}
  \item \textbf{kNN-граф:} Распределения числа компонент при $H_0$ и $H_1$ сильно перекрываются — низкая разделяющая способность, мощность маленькая.
  \item \textbf{DIST-граф:} Распределения кликового числа сдвинуты друг от друга: для SkewNormal значения пик около 50, для Student-t — около 39. Красная зона — область принятия H1: мощность лучше
\end{itemize}

\end{document}
