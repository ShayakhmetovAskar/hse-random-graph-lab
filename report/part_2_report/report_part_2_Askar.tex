% encoding: utf-8
\documentclass[12pt,a4paper]{article}
% поддержка русского
\usepackage[T2A]{fontenc}
\usepackage[utf8]{inputenc}
\usepackage[russian]{babel}
\usepackage{cmap}            % копирование кириллицы из PDF
% математические пакеты
\usepackage{amsmath,amssymb}
\usepackage[fleqn]{amsmath}  % выравнивание формул слева
% графика
\usepackage{graphicx}
\graphicspath{{./}}
% таблицы
\usepackage{booktabs}
\usepackage{array}
% гиперссылки
\usepackage[hidelinks]{hyperref}
% отступ первой строки абзаца
\setlength{\parindent}{1em}
% расстояние между абзацами
\setlength{\parskip}{0.5em}
% заголовок
\title{Отчет по части II}
\author{Шаяхметов Аскар}
\date{} % убрать дату

\begin{document}
\maketitle


\textbf{Гипотезы:}
\begin{itemize}
    \item $H_0$: данные из распределения skewnorm с параметром $\alpha = 1$
    \item $H_1$: данные из распределения student\_t с параметром $\nu = 3$
\end{itemize}

\textbf{Параметры исследования:}
\begin{itemize}
    \item Тип графа: dist-граф с параметром $d = 0.5$
    \item Размеры выборок: $n = 25, 100, 500$
    \item Количество выборок на класс: 500
\end{itemize}

\textbf{Исследуемые характеристики графов:}
\begin{itemize}
    \item $\Delta(G)$ --- максимальная степень вершины
    \item $\delta(G)$ --- минимальная степень вершины  
    \item $c(G)$ --- количество компонент связности
    \item $t(G)$ --- количество треугольников
    \item $\text{diam}(G)$ --- диаметр графа
    \item $\lambda(G)$ --- рёберная связность
    \item $\omega(G)$ --- кликовое число
\end{itemize}

\section{Результаты}

\subsection{Анализ важности характеристик}

Анализ важности характеристик с использованием Random Forest показал следующие результаты:\\

\includegraphics[width=1\linewidth]{feature_importance.png}
    

\textbf{Основные наблюдения:}
\begin{itemize}
    \item Для малых выборок ($n=25$) наиболее важной характеристикой является количество треугольников $t(G)$ (42.5\% важности)
    \item С ростом размера выборки важность максимальной степени $\Delta(G)$ увеличивается: от 17\% при $n=25$ до 29.6\% при $n=500$
    \item Минимальная степень $\delta(G)$ практически теряет значение с ростом $n$
\end{itemize}

\vfill

    
\subsection{Сравнение классификаторов}

Для оценки качества классификации использовались следующие алгоритмы: Random Forest, Logistic Regression и Neural Network. Результаты представлены на графике:

\includegraphics[width=1\linewidth]{comparison.png}

\textbf{Основные выводы по классификации:}
\begin{itemize}
    \item Для малых выборок ($n=25$) все классификаторы показывают умеренное ($\approx0.83$) качество с высокой ошибкой первого рода ($\alpha > 0.14$)
    \item При $n=100$ качество классификации резко улучшается, ошибка первого рода снижается до уровня ($\alpha \approx 0.01$)
    \item Для больших выборок ($n=500$) все классификаторы показывают практически идеальное качество
\end{itemize}

\subsection{Анализ распределений характеристик}
\includegraphics[width=0.7\linewidth]{dists.png}\\\\\\
Гистограммы распределений характеристик графов показывают четкое разделение между гипотезами $H_0$ и $H_1$ для нектороых характеристик.
\begin{itemize}
    \item Максимальной степени $\Delta(G)$ --- разделение улучшается при увеличении $n$
    \item Количества треугольников $t(G)$ --- четкое разделение для $n=500$
    \item Диаметра графа $\text{diam}(G)$ --- приемлемое разделение
    \item Кликового числа $\omega(G)$ --- для $n=500$ хорошее разделение
\end{itemize}
С увеличением размера выборки разделение между распределениями становится более выраженным, что объясняет улучшение качества классификации.

\section{Выводы}

Анализ результатов показал следующее:

\begin{itemize}
    \item \textbf{Для $n=25$}: ни один классификатор не удовлетворяет условию $\alpha \leq 0.05$
    \item \textbf{Для $n=100$}: лучший классификатор --- Random Forest с ошибкой первого рода $\alpha = 0.008$ и мощностью $0.991$
    \item \textbf{Для $n=500$}: лучший классификатор --- Neural Network (два скрытых слоя размерами 50 и 30) с ошибкой первого рода $\alpha = 0.000$ и мощностью $0.999$
\end{itemize}


\end{document}
